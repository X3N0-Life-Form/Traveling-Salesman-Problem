\documentclass[a4paper,10pt]{report}
\usepackage[utf8]{inputenc}
\usepackage[T1]{fontenc}
\usepackage{graphicx}
\usepackage[francais]{babel}
\usepackage{hyperref}
\usepackage{color}
\usepackage{algpseudocode}
\usepackage{algorithm}

% Title Page
\title{Opérateurs de recherche et stratégies d'exploration}
\author{Adrien DROGUET}
\date{Avril - Juin 2014}

\pagestyle{plain}

\begin{document}
\maketitle
%customiser page de titre
%ajouter nom de l'encadrant
%logo univ

\tableofcontents
\pagebreak

\chapter*{Introduction}
\section*{Définitions}
\begin{itemize}
 \item \textbf{Problème :} Ensemble de villes devant être parcourues.
 \item \textbf{Voisinage :} Chemin parcourant toutes les villes sans répétition.
Solution au problème du voyageur de commerce.
 \item \textbf{Voisin améliorant :} Chemin voisin d'une solution X, dont le coût
est inférieur au coût de X.
 \item \textbf{Recherche locale :} Exploration exhaustive de l'ensemble des
voisins d'une solution donnée.
 Coûteux en temps d'exécution.
 \item \textbf{Relation de voisinage :} Fonction permettant de passer d'un
voisinage à un autre.
 \begin{itemize}
  \item \textit{\textbf{Swap :}} Échange la position de deux villes dans un
chemin.
  \item \textit{\textbf{Insert :}} Insert une ville avant une autre.
  \item \textit{\textbf{Reverse :}} Inverse un sous section d'un chemin.
 \end{itemize}
 \item \textbf{Stratégie de sélection :} Détermine la manière dont on choisit un
voisin en recherche locale.
 \begin{itemize}
  \item \textit{\textbf{First Fit :}} Choisir le premier voisin améliorant
trouvé.
  \item \textit{\textbf{Best Fit :}} Choisir le meilleur améliorant trouvé.
  \item \textit{\textbf{Worst Fit :}} Choisir le moins bon améliorant trouvé.
 \end{itemize}
\end{itemize}


\pagebreak
\section*{Introduction}

\paragraph{} % sujet
  Le but de ce projet est d'étudier l'efficacité de différents opérateurs de
recherche et stratégies applicables à la résolution de ce problème en recherche
locale. Suite à l'étude des résultats de ces différents algorithmes et
stratégies, le projet vise la mise en place de méthodes permettant d'effectuer
une recherche partielle capable d'obtenir des résultats comparables à ceux d'une
recherche locale à un coût moindre en temps de calcul.
\paragraph{}
  Le voyageur de commerce est un problème bien connu dans le milieu informatique
: pour un ensemble de villes, minimiser la distance requise qu'un individu doit
parcourir afin de visiter chaque ville. De nombreuses études ont été menées sur
ce sujet et une quantité significative d'algorithmes de résolution ont été écrit
au fil des ans. Pour cette raison, cette problématique forme un cadre
intéressant pour l'analyse de l'efficacité des différents opérateurs
relationnels que l'ont cherche à étudier.

\paragraph{} % ressources utilisées
  Pour référence, nous avons utilisé les problèmes fourni par
\href{https://www.iwr.uni-heidelberg.de/groups/comopt/software/TSPLIB95/}{TSPLIB
}
- une bibliothèque d'échantillons d'instances pour le TSP (et autres problèmes
liés) provenant de diverses sources et de types variés. Dans notre cas, nous
nous sommes confiné à des problèmes symétriques à deux dimensions, avec des
distances entières ou réelles.


%%%%%%%%%%%%%%%%%%%%
% Partie Recherche %
%%%%%%%%%%%%%%%%%%%%
\part{Recherche}
\chapter{Recherche locale}
\section{Implémentation des algorithmes}
\subsection{Relations de voisinage}

\paragraph{}
  Après avoir établit une représentation satisfaisante et un système de lecture
des données fonctionnel (voir partie développement, sous-section représentation
des données), nous avons défini trois types de mouvements, ou relation de
voisinage à appliquer à une solution : l'échange (Swap), l'insertion (Insert) et
l'inversion (Reverse).

\paragraph{}
Afin d'éviter de recalculer le coût d'une solution (c'est à dire la distance que
le voyageur doit parcourir pour un chemin donné), à chacune de ces relations est
associé une fonction de calcul de coût potentielle. Celle-ci prend en paramètre
les villes sur lesquels seront appliqués les déplacements, et effectue les
calculs aux points de rupture du chemin modifié.
% insérer schéma types de déplacements + points de rupture
% reprendre comme exemple 0123456789


\subsection{Stratégies de sélection d'un voisinage}

\paragraph{}
  Une fois que l'on dispose d'une fonction permettant de calculer le voisinage
d'une solution donnée, il faut déterminer une stratégie d'amélioration à
appliquer. Nous avons retenu trois stratégies distinctes à appliquer à notre
recherche locale : choisir le premier voisin améliorant que l'on trouve (First
Fit), choisir le meilleur voisin améliorant après avoir parcouru l'ensemble des
voisins possibles (Best Fit), ou bien choisir le voisin améliorant le moins
possible la solution actuelle (Worst Fit).
% schémas d'exemples


\subsection{Lancement}

\paragraph{}
  Suite au parsage d'un TSP particulier, notre programme s'efforce d'appliquer
chacune des relations et chacune des stratégies spécifiées l'une après l'autre,
en partant éventuellement du même point de départ. Le voisinage de départ est
généré aléatoirement à partir des données du problème. Chaque pas est exploré de
manière aléatoire, bien que dans le cas des stratégies \textit{Best Fit} et
\textit{Worst Fit}, on va chercher (dans un premier temps) à explorer l'ensemble
des voisins possibles.

\paragraph{}
En fin d'exécution, un résumé des résultats obtenus est affiché en console.

\begin{figure}[h]
Run Results for a280:
 \begin{center}
Run Data: Relation=Reverse;	Strategy=First Fit;	starting cost=34194,
  end cost=3026;	depth=1182;	run time= 0.448 sec\linebreak
Run Data: Relation=Reverse;	Strategy=Best Fit;	starting cost=34194,
  end cost=2935;	depth=297;	run time= 9.652 sec\linebreak
Run Data: Relation=Reverse;	Strategy=Worst Fit;	starting cost=34194,
  end cost=3010;	depth=29699;	run time= 22 min 32.312 sec\linebreak
 \end{center}
  \label{a280-sample-results}
  \caption{Résumé de fin d'exécution - format texte}
\end{figure}

\paragraph{}
Instances utilisées (par ordre de lancement) :
\begin{itemize}
  \item \textbf{a280.tsp} : Instance principalement utilisée par les tests
unitaires, d'une taille respectable, mais pas excessive.
  \item \textbf{att48.tsp} : Instance de petite taille, utilisée lors de tests
manuels rapide ou en débuggage.
  \item \textbf{berlin52.tsp} : Instance de petite taille.
  \item \textbf{ali535.tsp} : Instance de grande taille, beaucoup utilisée
étalon lorsque l'on cherche à améliorer les temps d'exécution.
  \item \textbf{ch130.tsp} : Instance de taille moyenne.
  \item \textbf{ch150.tsp} : Instance de taille moyenne.
  \item \textbf{burma14.tsp} : Instance de très petite taille.
  \item \textbf{bier127.tsp} : Instance de taille moyenne, parfois substituée à
a280 dans les tests.
  \item \textbf{brd14051.tsp} : Instance de taille excessive, seule la stratégie
\textit{First Fit} met un temps "raisonnable" à atteindre un optimum local.
\textit{Best Fit} et \textit{Worst Fit} n'ont jamais été testé sur cette
instance.
\end{itemize}

\pagebreak
\section{Analyse et comparaison de l'efficacité des algorithmes}
\subsection{Enregistrement des résultats}

\paragraph{}
  Après chaque exécution, on en enregistre les résultats dans des fichiers sous
forme textuelle. On distingue deux types de fichiers de sortie. D'une part les
*.results sont équivalents à la sortie console en fin de programme et
fournissent un résumé facile à lire pour les humain.
\paragraph{}
  D'autre part, les *.csv contiennent les mêmes données enregistrées au format
Comma Separated Value, ie. valeurs séparées par virgule, chaque ligne
correspondant à un couple Relation/Stratégie. Typiquement, le contenu de ces 
fichiers est concatené dans un fichier résumé, qui peut alors être copié tel
quel dans un tableur type Microsoft Excel ou Libre Office Calc. Voir en annexe
pour description plus approfondie du tableur utilisé.

\begin{figure}[h]
  \begin{center}
    \begin{tabular}{cccrrrr}
      a280, &Reverse, &First Fit, &34194, &3026, &1182, &0.448\\
      a280, &Reverse, &Best Fit,  &34194, &2935, &297,  &9.652\\
      a280, &Reverse, &Worst Fit, &34194, &3010, &29699,&1352.312\\
    \end{tabular}
  \end{center}
  \label{a280-sample-csv}
  \caption{Résumé de fin d'exécution - format .csv}
\end{figure}

\subsection{Efficacité constatée}

\begin{figure}[h]
  \begin{center}
    \begin{tabular}{ccrrr}
      \textbf{Opérateur}& \textbf{Stratégie}& \textbf{Coût}& \textbf{Tps
d'exécution}& \textbf{Profondeur}\\
%     Opérateur&	Stratégie&	Coût Moyen&	Temps d'exécution&
%Profondeur\\
      Swap&	First Fit&	35 404,39&	107,186&	150 717\\
      Swap&	Best Fit&	37 313,41&	1 198,021&	46 680\\
      Swap&	Worst Fit&	33 743,71&	63 440,385&	3 275 847\\
      Insert&	First Fit&	28 531,77&	55,669&		144 553\\
      Insert&	Best Fit&	29 643,06&	233,594&	22 209\\
      Insert&	Worst Fit&	28 420,68&	65 041,299&	4 667 846\\
      Reverse&	First Fit&	25 823,20&	88,784&		148 971\\
      Reverse&	Best Fit&	25 344,58&	409,759&	16 198\\
      Reverse&	Worst Fit&	26 349,78&	58 323,618&	3 912 314\\
    \end{tabular}
  \end{center}
  \label{recap-general}
  \caption{Exemple de données générales recueillies après 40 exécutions sur
chaque instance}
\end{figure}

% résultats spécifiques

\begin{figure}[ht]
  \begin{center}
    \begin{tabular}{|l|r|r|r|}
      \hline
      &		\textbf{First Fit}&	\textbf{Best Fit}&	\textbf{Worst
Fit}\\\hline
      \textbf{Swap}&	
	  7 653,780&
	  8 516,976&
	  \textbf{\textcolor{blue}{6 437,415}}\\\hline
      \textbf{Insert}&
	  5 435,000&
	  5 674,415&
	  \textbf{\textcolor{blue}{4 969,829}}\\\hline
      \textbf{Reverse}&
	  4 508,780&
	  \textbf{\textcolor{blue}{4 346,805}}&
	  4 532,195\\\hline
    \end{tabular}
    \label{a280-results}
    \caption{Résultats moyens pour le TSP a280}
  \end{center}
\end{figure}

\begin{figure}[ht]
  \begin{center}
    \begin{tabular}{|l|r|r|r|}
      \hline
      &		\textbf{First Fit}&	\textbf{Best Fit}&	\textbf{Worst
Fit}\\\hline
      \textbf{Swap}&
	  3 200,600&
	  3 432,550&
	  \textbf{\textcolor{blue}{2 662,600}}\\\hline
      \textbf{Insert}&
	  2 516,450&
	  3 299,950&
	  \textbf{\textcolor{blue}{2 359,950}}\\\hline
      \textbf{Reverse}&
	  \textbf{\textcolor{blue}{861,250}}&
	  861,350&
	  861,400\\\hline
    \end{tabular}
    \label{ali535-results}
    \caption{Résultats moyens pour le TSP ali535}
  \end{center}
\end{figure}

\begin{figure}[ht]
  \begin{center}
    \begin{tabular}{|l|r|r|r|}
      \hline
      &		\textbf{First Fit}&	\textbf{Best Fit}&	\textbf{Worst
Fit}\\\hline
      \textbf{Swap}&
	  53 327,317&
	  55 026,098&
	  \textbf{\textcolor{blue}{51 228,220}}\\\hline
      \textbf{Insert}&
	  44 155,293&
	  44 656,390&
	  \textbf{\textcolor{blue}{43 567,049}}\\\hline
      \textbf{Reverse}&
	  40 788,024&
	  \textbf{\textcolor{blue}{40 212,732}}&
	  40 826,512\\\hline
    \end{tabular}
    \label{att48-results}
    \caption{Résultats moyens pour le TSP att48}
  \end{center}
\end{figure}

\begin{figure}[ht]
  \begin{center}
    \begin{tabular}{|l|r|r|r|}
      \hline
      &		\textbf{First Fit}&	\textbf{Best Fit}&	\textbf{Worst
Fit}\\\hline
      \textbf{Swap}&
	  \textbf{\textcolor{blue}{24,683}}&
	  25,756&
	  26,098\\\hline
      \textbf{Insert}&
	  26,098&
	  \textbf{\textcolor{blue}{24,585}}&
	  24,732\\\hline
      \textbf{Reverse}&
	  \textbf{\textcolor{blue}{23,659}}&
	  23,707&
	  24,390\\\hline
    \end{tabular}
    \label{burma14-results}
    \caption{Résultats moyens pour le TSP burma14}
  \end{center}
\end{figure}

\paragraph{}
Note: les résultats des autres TSP sont disponibles en annexe.

\paragraph{}
  Comme on peut le constater, l'opérateur \textbf{\textit{Reverse}} est
nettement plus efficace que \textbf{\textit{Swap}} et \textbf{\textit{Insert}}.
En terme de stratégie, \textbf{Worst Fit} est la plus efficace combinée avec les
opérateurs \textbf{\textit{Swap}} et \textbf{\textit{Insert}}, tandis que
\textbf{Best Fit} est la stratégie de choix pour \textbf{\textit{Reverse}}.

\chapter{Analyse d'efficacité et espérance d'amélioration}

\paragraph{}
  Suite à l'étude préliminaire de l'efficacité de chacun des relations et
stratégies, on a cherché à étudier plus en détail le comportement de ces
algorithmes en cours d'exécution. Nous nous sommes concentré sur l'opérateur
relationnel le plus efficace sur l'ensemble de nos résultats :
\textbf{\textit{Reverse}}. La première étape est l'enregistrement de l'exécution
du programme pas à pas.

\section{Enregistrement de l'exécution}
\subsection{Intégration dans le code existant}

\paragraph{}
  Avant toute chose, il a fallu concevoir de nouvelles structures capable de
représenter l'état de notre programme à un instant \textit{t}. Ensuite, ces
classes ont été greffées au code existant sans en modifier le comportement.
Finalement, nous avons choisi d'enregistrer chaque étape, chaque ``pas'' ligne
par ligne dans un fichier texte. Une fois encore, nous utilisons le format .csv
afin de pouvoir aisément transférer ces données dans un tableur au besoin.

\paragraph{}
  Via le tableur, nous avons pu établir des graphe retraçant l'historique de
chaque exécution. La contrepartie de cette méthode est la relative difficulté à
comparer de nombreux résultats cote à cote, le logiciel n'étant pas bien adapté
pour traiter de tels volumes de données de manière dynamique.

\subsection{Données enregistrées}

\paragraph{}
  Dans un premier temps, nous avons classé les opérations de type
\textit{Reverse} en fonction de leur portée, c'est à dire le nombre d'éléments
(villes) affectés. On détermine ainsi de manière dynamique une série
d'intervalles de taille croissante pour chacune de nos instances. Nous avons mis
en place deux façon de créer nos intervalles : soit ils sont disjoint (voir
tableau ci-dessous), avec la borne inférieure égale à la borne supérieure de
l'intervalle précédent ; soit nous leur attribuons à tous la borne inférieure 1.
Le type et le taux de croissance peuvent être spécifiés au lancement de
l'application.

\begin{figure}[h]
  \begin{center}
    \begin{tabular}{cccccc}
      Reverse,&First Fit,&[1;4[,&[4;16[,&[16;64[,&[64;281[,\\
      Reverse,&First Fit,&0,    &0,     &1,      &1,\\
      Reverse,&First Fit,&0,    &0,     &2,      &1,\\
      Reverse,&First Fit,&0,    &0,     &2,      &2,\\
      Reverse,&First Fit,&0,    &0,     &2,      &3,\\
      Reverse,&First Fit,&0,    &0,     &2,      &4,\\
      Reverse,&First Fit,&0,    &0,     &2,      &5,\\
      Reverse,&First Fit,&0,    &0,     &3,      &5,\\
      Reverse,&First Fit,&0,    &0,     &3,      &6,\\
      Reverse,&First Fit,&0,    &0,     &3,      &7,\\
      Reverse,&First Fit,&0,    &0,     &3,      &8,\\
    \end{tabular}
  \end{center}
  \label{a280-sample-interval-simple}
  \caption{Extrait simplifié d'historique d'exécution - statistiques générales}
\end{figure}

\paragraph{}
  Dans un second temps, nous enregistrons les informations ayant trait à
l'exploration du voisinage et à l'action choisie. Ainsi, nous prenons note de
la portée de l'opération, de son degré d'amélioration (de combien le coût a
diminué), avec la quantité de voisins améliorants et de voisins dégradant
trouvés, accompagné du ratio voisins améliorants/voisins explorés.

\begin{figure}[h]
  \begin{tabular}{lcccl}
    Range,		&Cost diff,	
				&Stats,	&improv/degrad actions	&ratio,\\
    (23  ; 143),120,	&104,	&Stats:,	&1,2,		&0.333333,\\
    (117 ; 26),  59,	&788,	&Stats:,	&1,4,		&0.2,\\
    (117 ; 14),  47,	&106,	&Stats:,	&1,2,		&0.333333,\\
    (137 ; 81),  94,	&322,	&Stats:,	&1,1,		&0.5,\\
    (132 ; 22),  40,	&466,	&Stats:,	&1,7,		&0.125,\\
    (137 ; 81),  94,	&600,	&Stats:,	&1,1,		&0.5,\\
    (82  ; 86),   4,	&108,	&Stats:,	&1,6,		&0.142857,\\
    (1   ; 14),  13,	&22,	&Stats:,	&1,8,		&0.111111,\\
    (25  ; 129),104,	&548,	&Stats:,	&1,10,		&0.0909091,\\
    (91  ; 80), 139,	&662,	&Stats:,	&1,0,		&0,\\
    (137 ; 81),  94,	&94,	&Stats:,	&1,1,		&0.5,\\
    (100 ; 50), 100,	&334,	&Stats:,	&1,8,		&0.111111,\\
  \end{tabular}
  \label{a280-sample-interval-simple-bis}
  \caption{Extrait simplifié d'historique d'exécution - détails actions}
\end{figure}


\pagebreak
\section{Analyse des résultats}

%insert graphs

\section{Espérance d'amélioration - Implémentation d'un système de décision}

\paragraph{}
  Grâce aux résultats obtenus, il est possible d'analyser plus en profondeur le
comportement de nos opérateurs et stratégie, et ainsi déterminer un système
permettant d'effectuer les recherches de manière plus intelligente. L'idée
directrice est d'avoir un ensemble de dispositifs permettant de jauger la
pertinence d'un voisinage donné en fonction des voisinages précédemment choisis.
%rechercher de manière intelligente / réduire espace de recherche

\begin{figure}[h]
  \begin{tabular}{lll}
    Run Data: Relation=Reverse;&
      Strategy=First Fit;\\
      &starting cost=157894,&
      end cost=34817;\\
      &depth=135;\\
      &run time= 0.004 sec\\
    Run Data: Relation=Reverse;&
      Strategy=Best Fit;\\
      &starting cost=157894,&
      end cost=35155;\\
      &depth=46;\\
      &run time= 0.036 sec\\
    Run Data: Relation=Reverse;&
      Strategy=Worst Fit;\\
      &starting cost=157894,&
      end cost=36260;\\
      &depth=5374;\\
      &run time= 3.913 sec\\
  \end{tabular}
  \caption{Résultats témoins}
\end{figure}


\subsection{Piste 1 : Probabilité de considérer un voisin}
\subsubsection{Idée}

\paragraph{}
À chaque intervalle est associé une probabilité. A chaque voisin est associé un
intervalle. Nous sélectionnons désormais nos voisins avec l'algorithme suivant :

\begin{algorithm}[h]
  \begin{algorithmic}
    \ForAll{voisins}
      \State $roll\gets \Call{valeurAleatoire}{}$
      \If{$roll \leq voisin.intervalle.probabilite$}
	\State $selection \gets voisin$
	\State break
      \EndIf
    \EndFor
    \If{\Call{satisfaitStrategie}{selection}}
      \Return selection
    \EndIf
  \end{algorithmic}
  \caption{Sélection probabiliste}
\end{algorithm}

\paragraph{}
\underline{Note :} La variable \textit{selection} peut avoir été modifiée par la
fonction \textit{satisfaitStrategie}, en fonction de la stratégie utilisée.

\paragraph{}
  On espère ainsi qu'une fois arrivé en milieu ou fin de recherche, on ignorera
des mouvements que l'on estime avoir peu de chances d'améliorer la solution
actuelle. La probabilité d'un intervalle évolue en fonction du nombre de voisins
acceptants qu'il produit.

\pagebreak
\begin{algorithm}[h]
  \begin{algorithmic}
    \Require $selection$ : Voisin
    \Require $increment > 0$
    \State 
    \Comment{Voir sous-section \textit{Problèmes} ci-dessous pour la description
du calcul de l'incrément}
    \State $probabilite \gets selection.intervalle.probabilite$
    \If $ameliorant(selection)$
      \State $selection.intervalle.probabilite \gets probabilite + increment$
    \Else
      \State $selection.intervalle.probabilite \gets probabilite - increment$
    \EndIf
    \State ...
    \Comment{Suite de la procédure dépend du type de stratégie choisie}
    \Ensure $selection.intervalle.probabilite \leq 1$
  \end{algorithmic}
  \caption{satisfaitStrategie (Générique)}
\end{algorithm}

\paragraph{}
\underline{Note :} On modifie la probabilité d'un seul intervalle à la fois.
Alternativement, on peut choisir de n'effectuer qu'une seule condition de
changement de la probabilité (décroître si voisin non-améliorant).

%insérer résultats


\subsubsection{Problèmes}

\paragraph{}
  Dans notre cas, l'obtention d'un jet aléatoire prend plus de temps processeur
que de calculer le coût d'un voisin potentiel. On aura également des difficultés
à faire en sorte que les probabilités entre les différents intervalles soient
significatives sans évoluer trop rapidement. Une évolution trop rapide peut nous
amener à refuser tous les voisins d'une solution à une profondeur relativement
faible, ce qui peut s'avérer particulièrement problématique dans le cas de
\textit{Worst Fit}.
\paragraph{}
  Pour remédier en partie à cela, la formule d'évolution de la probabilité prend
en compte la taille du problème à traiter et la différence de coût par rapport à
la solution précédente.

\pagebreak
\subsection{Piste 2 : Trier les intervalles}
\subsubsection{Idée}

\paragraph{}
  Étant donné que, d'après nos résultats d'exécution, les grands intervalles ont
tendance à être beaucoup plus intéressants en début et milieu de recherche, nous
avons cherché à orienter le chemin de la recherche. En modifiant les
probabilités de départ de chaque intervalle en fonction de leur taille (plus
grands = probabilité plus importante), on espère limiter le nombre d'analyses
``inutiles'' en début de recherche.

%relancer quelques run (cf TODO list)
%faire algo
%insérer résultats

\subsubsection{Problèmes}

\paragraph{}
  Une fois encore, on se heurte au coût des jets aléatoires. La problématique
de l'évolution des coûts reste présente.


\pagebreak
\subsection{Piste 3 : Échantillonnage acceptable}
Note: Non implémenté.
\subsubsection{Idée}
%probabilité d'arrêter la recherche en fonction du nombre de voisins explorés et
%de passer au niveau suivant

\subsubsection{Problèmes potentiels}


%%%%%%%%%%%%%%%%%%%%
% Partie technique %
%%%%%%%%%%%%%%%%%%%%
\part{Développement}
\chapter{Architecture de l'application}
\section{Structures de base}
\subsection{Représentation des données}
%core + parsage
\paragraph{}
  Les .tsp fournis par LIBTSP suivent assez rigoureusement la même structure de
fichier, ce qui les rend facile à lire. Étant donné que nous n'auront
vraisemblablement qu'un seul problème en mémoire à un instant donné, on peut se
permettre d'être assez gourmand. Les objets \textbf{Problem} et \textbf{City}
forment le cœur de nos données et nous servent à faire abstraction de certaines
caractéristiques des différents TSP, principalement de la représentation des
coordonnées (nombres entiers ou réels). Voir la section \textit{core} en annexe
pour plus de détails.

\subsection{Compartimentalisation des algorithmes}
%relation & stratégies
\paragraph{}
  Afin de séparer au mieux les différents opérateurs relationnels et stratégies
tout en évitant autant que possible les duplications de code, nous avons tiré
avantage du polymorphisme et de l'héritage pour généraliser le traitement de ces
objets. Voir les diagrammes de classe dans les sections \textit{relation} et
\textit{strategy} en annexe.


\section{Exécution}
\subsection{Lancement du programme}
%main, parser & runner
\paragraph{}
  Le programme s'exécute en ligne de commande avec un certain nombre d'arguments
obligatoires, tels que le nom du fichier .tsp ou les relations de voisinage et
stratégies à utiliser, et des arguments optionnels activant out désactivant
certains contrôles ou écritures de trace. Une fois tous les objets créés et
configurés, le main passe la main à la classe \textit{Runner} (voir annexe).

\subsection{Analyse Comportementale}
%interval stuff
\paragraph{}
  On utilise 3 classes afin d'enregistrer chaque pas de l'exécution de notre
programme: \textit{Action}, \textit{Interval} et \textit{IntervalManager}.
\textit{Action} mémorise quel voisin a été choisi et avec quel gain (ou perte)
par rapport à la solution actuelle. \textit{Interval} permet de classer ces
actions et \textit{IntervalManager} sert de chef d'orchestre. Du fait des
informations mises en mémoire par ces classes, les entités dédiées à la
recherche partielle en font également usage pour déterminer la pertinence d'un
voisinage.

\subsection{Recherche partielle}
%choice & hook
\paragraph{}
  L'organisation des classes de prise de décision est légèrement plus chaotique.
La classe \textit{ChoiceMaker} correspond à la première tentative d'analyse de
pertinence. Elle implémente l'interface \textit{Hookable} afin de s'interfacer
avec les classes de type \textit{Relation} sans causer de conflits d'inclusion
multiples. \textit{ChoicePicker} a un rôle similaire, implémente la même
interface et, du point de vue de \textit{Relation}, opère de la même manière
que \textit{ChoiceMaker}.


\chapter{Déroulement du projet}
\section{Organisation}
\subsection{Contrôle d'avancement}

\subsubsection{Gestion de version}
\paragraph{}
  Afin de pouvoir garder un contrôle optimal sur les changements et évolutions
du projet, un système de contrôle de version Git, avec un dépôt distant hébergé
sur Github, a été adopté. Nous avons employé une approche type "commits
courants, commits courts", de diviser nos changements au mieux - ce qui facilite
les réversions lorsque l'on en a besoin.

\paragraph{}
  Combiné à un journal de bord personnel, l'historique des commits permet
d'avoir une vision d'ensemble précise de l'évolution de l'application.

\subsubsection{Rapports hebdomadaires et réunions}
\paragraph{}
  Chaque semaine, un compte-rendu de l'avancement, des difficultés rencontrées,
résultats obtenus et des évolutions à venir a été rédigé, accompagné de
réunions à chaque fois qu'une étape importante était atteinte - implémentation
des opérateurs et stratégie, des traces d'exécution, de la prise de décision...


\subsection{Contrôle de non-régression}
%tests unitaires
\paragraph{}
  La quasi-totalité des méthodes, fonctionnalités et bugs ont été contrôlés par
une batterie de tests unitaires, mettant parfois en évidence une régression ou
un comportement incorrecte. Ainsi, à chaque révision majeure ou en fin de
journée, on lance l'ensemble des tests pour vérifier que rien n'a cassé (y
compris parfois les tests eux-mêmes).

%retro compatibilité
\paragraph{}
  On s'est également efforcé de maintenir un certain degré de rétrocompatibilité
: la plupart des nouvelles méthodes ou algorithmes doivent être activés lors de
l'appel au programme en ligne de commande. Heureusement, le script de lancement
permet de maintenir une bonne utilisabilité de l'application malgré l'inflation
du nombre de paramètres optionnels.

\subsection{Profiling et Optimisation}
%valgrind, callgrind & autre
\paragraph{}
  Dans le but d'améliorer la vitesse d'exécution, l'utilisation mémoire de
l'application, et corriger des problèmes potentiels, nous avons tiré avantage
des capacités des \textbf{valgrind}, \textbf{callgrind} ainsi qu'une poignée de
contrôles manuels pour obtenir une image précise du comportement de notre
programme. Ces informations nous ont ensuite permis de commencer à optimiser
certaines sections du code.

\paragraph{}
  Cette opération a donné naissance à certaines structures allégées, comme des
voisins simplifiés, ne contenant que les informations essentielles aux classes
dont elles sont attributs. On s'est également efforcé de libérer dès que
possible les données ayant dépassé leur période d'utilité, rendant le programme
beaucoup plus robuste et capable de tenir la charge sur de grandes instances.


\section{Difficultés techniques}

\chapter*{Conclusion}
\section*{} %conclusion / rétrospective


\section*{Informations complémentaires}
%bibliographie
%licence
%repo github


\end{document}          
