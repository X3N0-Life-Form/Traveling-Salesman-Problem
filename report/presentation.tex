\documentclass{beamer}

\usepackage[utf8]{inputenc}
\usepackage{default}
\usepackage{graphicx}
\usepackage[francais]{babel}
\usepackage{algpseudocode}
\usepackage{algorithm}
\usepackage{mathtools}

% Title Page
\title{Relations de voisinage et stratégies d'exploration}
\author{Adrien DROGUET}
\date{Avril - Juin 2014}


\begin{document}

\frame{\titlepage}


\begin{frame}
  \frametitle{Étude de permutations et stratégies}
  \framesubtitle{Cadre - Recherche locale}

  % à l'oral: définir voisinage
  \begin{definition}
    Recherche d'une solution localement optimale (solution n'ayant pas de voisin
immédiatement améliorant) dans un espace de recherche donné.
  \end{definition}

  
\end{frame}


\begin{frame}
  \frametitle{Étude de permutations et stratégies}
  \framesubtitle{Voisinage}
  
  \begin{definition}
    Le \textbf{voisinage} d'une solution est l'ensemble des solutions se
trouvant à une distance de 1 permutation.
  \end{definition}

  \begin{definition}
    Une \textbf{relation de voisinage} est une opération appliquée à une
solution (permutation) permettant d'obtenir un voisinage de cette solution.
  \end{definition}
\end{frame}


\begin{frame}
  \frametitle{Étude de permutations et stratégies}
  \framesubtitle{Stratégies d'exploration}
  \begin{definition}
    On appelle \textbf{stratégie} le critère d'amélioration de notre recherche
locale.
  \end{definition}

  
  Notre système distingue trois types de stratégies:
  \begin{itemize}
   \item \textbf{First Fit} : Choisir le premier améliorant trouvé.
   \item \textbf{Best Fit} : Choisir le meilleur améliorant.
   \item \textbf{Worst Fit} : Choisir le pire améliorant.
  \end{itemize}

\end{frame}


\begin{frame}
  \frametitle{Étude de permutations et stratégies}
  \framesubtitle{Cadre - Problème du voyageur de commerce}
  \begin{definition}
    \textbf{Problème du voyageur de commerce} : pour un ensemble de villes
séparées par des distances données, trouver le chemin le plus court pour
parcourir toutes ces villes.
  \end{definition}
  %présentation du problème \& explication pourquoi c'est bien
  $==>$ Problème d'optimisation bien connu et étudié.
  
\end{frame}



\begin{frame}
  \frametitle{Phase 1 - Implémentation}
  
  
\end{frame}


\begin{frame}
  \frametitle{Phase 1 - Implémentation}
  \framesubtitle{Lancement et mesures préliminaires}
  
  \begin{center}
  \includegraphics[width=0.6\textwidth,height=0.75\textheight]{images/exec-phase-1.png}
  \end{center}
  
\end{frame}



\begin{frame}
  \frametitle{Phase 2 - Collecte d'information}
  
  \begin{center}
    \includegraphics[width=\textwidth,height=0.8\textheight]{images/exec-phase-2.png}
  \end{center}
  
\end{frame}


\begin{frame}
  \frametitle{Phase 2 - Collecte d'information}
  \framesubtitle{Analyse}
  
  \begin{itemize}
    \item \textbf{Relations de voisinage} :
      \begin{itemize}
	\item Inversion en tête
	\item Échange en dernière position
      \end{itemize}
    \item \textbf{Stratégies d'exploration} :
      \begin{itemize}
	\item Choisir le moins bon améliorant pour des voisinages d'échange et
d'insertion.
	\item Choisir le meilleur améliorant pour un voisinage d'inversion.
      \end{itemize}
  \end{itemize}
  
\end{frame}


\begin{frame}
  \frametitle{Phase 3 - Amélioration du système}
  
  
\end{frame}



\begin{frame}
  \frametitle{Conclusion}
  
  
\end{frame}


\begin{frame}
  \frametitle{Remerciements}
  
  
\end{frame}


\end{document}
